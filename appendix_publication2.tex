\onehalfspacing
\chapter*{Appendix B: Supporting information for Chapter \ref{ch:publication2}}


\noindent\textbf{Contents of this file}
\begin{enumerate}
\item Text S1 to S3
\item Figures S1 to S4
\end{enumerate}



\noindent\textbf{Introduction}
S1 contains a more complete discussion of clustering images, S2 and Figure S3 describe a synthetic example with a lower magnitude signal than used in the main text, S3 describes the attached animations, Figure S1 shows all the interferograms in our Sentinel-1 time series, Figure S2 shows the displacements recorded by several GPS stations surrounding Sierra Negra, and Figure S4 shows the probability density functions for the parameters estimated by our Bayesian inversion. 

\section*{Text S1}

When clustering data it can be intuitive to think of each sample of our dataset as a vector with a length determined by the number of variables we have.  Each sample can then be thought of as a point in a space with as many dimensions as we have variables and, our samples may form clusters that our algorithm of choice can identify.  In the case of a dataset with $100$ samples of three random variables, these clusters could be visualised in relative ease in a $3$D space.  However, when we wish to perform clustering with images (as are the output of sICA), our samples have as many variables as pixels, which in the case of interferograms tends to be  of the order $\sim10^{4}$.  Consequently, we have relatively few samples in a very high dimensional space, and clustering becomes difficult.  Therefore, clustering with images generally requires a specialised distance metric instead of using measures such as the Euclidean distance.  

The ICASO algorithm uses the absolute value of the correlation between source pairs as a similarity measure.  Through taking the absolute value, sign flipped versions of the same source attain a high similarity and do not form duplicate clusters.  A trivial step can be performed to convert similarities to distances (e.g. $\textbf{D} = 1 - \textbf{S}$, where $\textbf{D}$ is the distance matrix, and $\textbf{S}$ is our similarity matrix).  


\section*{Text S2}

Figure S3  displays the results of applying our automatic detection algorithm to a time series similar to that used in Section $2$ which features a more subtle signal and requires tuning of the frequency with which the lines of best fit are redrawn.  The new signal which enters the time series from interferogram $90$ onwards is of very small magnitude, and significant deviations from the lines of best fit are only observed for the RMS cumulative residual when the lines of best fit are redrawn every $60$ interferograms, and not in the case when they are redrawn every $20$ interferograms.  However, in the case that the lines of best fit are redrawn infrequently, the algorithm flags several events erroneously (``false positive" results), such as the peaks of IC3's time course.  We believe this demonstrates the importance of the correct configuration of the parameter, as our algorithm's sensitivity to small signals is achieved at the expense of an increased likelihood of false positive results.  Further use of our algorithm during its future application to other volcanoes is likely to shed further light on this issue, but remains beyond the scope of this initial study.  

It should also be noted that the sinusoidal trend in IC3's time course is due to the strength of our synthetic topographically correlated APS varying seasonally.  The less sensitive case in which the lines of best fit are redrawn every $30$ interferograms is more successful in avoiding the false positives that are seen when the lines are redrawn every $60$ interferograms.  This seasonal variation is an intrinsic part of our the nature of the baseline data, and through fitting a linear trend we fail to accurately characterise it.  Therefore, we postulate that in future use, more complex functions may allow us to characterise the temporal nature of certain atmospheric signals, and so ultimately increase the sensitivity of the detection algorithm.  

\section*{Text S3}

Implementing ICA  to recover statistically independent sources is very closely related to applying PCA to recover uncorrelated sources.  Informally, two signals are statistically independent if knowledge of the value of one signal does not convey any information of the value of the other, which contrasts with correlation that merely measures the linear relationship between variables.  A similar measure applied to signals is that of correlation, and two signals are said to be uncorrelated if their covariance is $0$.  The difference between uncorrelatedness and the more important statistical independence can be demonstrated with two random variables that, when plotted in 2d space, form a circle around the origin.  Whilst the correlation of these values is 0, it is clear that they are not statistically independent as knowledge of one signal conveys information about the second signal (e.g. signal 1 attaining its maximum value conveys the information that signal 2 must also be at its median value).  Therefore, two uncorrelated variables can still provide information about each other.  The wealth of successful applications of ICA to BSS problems can be used to justify the expectation that it would outperform PCA as two physical processes that are unrelated (such as deformation at a volcano and atmospheric delay) are statistically independent, as opposed to merely uncorrelated.  


\begin{figure}[]
	\centering 
 	\includegraphics[width=1\textwidth]{./publication2_figures/outputs_png/figure_s1.png}
	\caption[Sierra Negra interferograms]{Time series of Sentine-1 interferograms ($0-97$) and the corresponding DEM.   }
	\label{fig:s1}
\end{figure}

\begin{figure}[]
	\centering 
 	\includegraphics[width=1\textwidth]{./publication2_figures/outputs_png/figure_S3_gps_stations.png}
	\caption[Sierra Negra GPS data]{GPS data spanning the $2005 - 2018$ inter-eruptive period.  For each station (GV01, GV04, GV06, and GV09) the top plot shows the cumulative displacement in either East/North/Up directions, whilst the lower plot shows the ratios of each possible pair of directions (e.g. East vs. North).  With the exception of a brief period around day $2500$, the ratios remain constant, which we conclude shows that the style of deformation observed during the Sentinel-1 time series is likely to have remained similar for the entire inter-eruptive period.    }
	\label{fig:s2}
\end{figure}

\begin{figure}[]
	\centering 
 	\includegraphics[width=0.8\textwidth]{./publication2_figures/outputs_png/figure_S4.png}
	\caption[Automatic detection algorithm: Low magnitude signal]{Application of our automatic detection algorithm to a synthetic time series similar to that presented in Figure $3$.  The upper and lower halves of the figure show the effects of redrawing the lines of best fit every $20$ and $60$ interferograms, respectively. The first $90$ interferograms contain signals from a topographically correlated APS (recovered as IC3), an east-west phase gradient (recovered as IC2), and a turbulent APS.  The remaining interferograms ($90 - 198$) contain a small synthetic deformation signal in the centre of the frame which is difficult to identify by eye.  However, as we are unable to fit this new signal with the learned components, the RMS cumulative residual increases in slope, but this is only detected (orange and yellow highlighting of points) when the lines of best fit are redrawn every $60$ interferograms. }
	\label{fig:s3}
\end{figure}


\begin{figure}[]
	\centering 
 	\includegraphics[width=0.8\textwidth]{./publication2_figures/outputs_png/figure_S5.png}
	\caption[Bayesian inversion parameters]{Results of our Bayesian inversion for our variable opening constant pressure horizontal dislocation. Red lines indicate the optimal values, which we report in the main text. }
	\label{fig:s4}
\end{figure}

